%%
%% The "title" command has an optional parameter, 
%% allowing the author to define a "short title" to be used in page headers.
\title{Estimating the Impact of GitHub Copilot on Developer Productivity Using Italy's ChatGPT Ban as Natural Experiment}

%%
%% The "author" command and its associated commands are used to define
%% the authors and their affiliations.
%% Of note is the shared affiliation of the first two authors, and the
%% "authornote" and "authornotemark" commands
%% used to denote shared contribution to the research.
\author{Richard Schulz}
\authornote{Both authors contributed equally to this research.}
\email{rschul@ethz.ch}
\affiliation{%
  \institution{ETH Zurich}
  \city{Zurich}
  \country{Switzerland}
}

\author{Simon Schuhmacher}
\authornotemark[1]
\email{simon.schuhmacher@uzh.ch}
\affiliation{%
  \institution{University of Zurich}
  \city{Zurich}
  \country{Switzerland}
}

%%
%% By default, the full list of authors will be used in the page
%% headers. Often, this list is too long, and will overlap
%% other information printed in the page headers. This command allows
%% the author to define a more concise list
%% of authors' names for this purpose.
% \renewcommand{\shortauthors}{Trovato et al.}

%%
%% The abstract is a short summary of the work to be presented in the
%% article.

% "You can omit the abstract to save on space,
% just make sure to motivate your work in the introduction."
% \begin{abstract}
%   Abstract
% \end{abstract}

%%
%% This command processes the author and affiliation and title
%% information and builds the first part of the formatted document.
\maketitle
